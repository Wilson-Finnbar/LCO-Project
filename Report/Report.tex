\documentclass[11pt]{article}
\usepackage{graphicx}
\usepackage{array}
\usepackage{booktabs}
\usepackage[a4paper,margin=3cm, headheight=90pt]{geometry}
\usepackage{cmbright}
\usepackage[OT1]{fontenc}
\usepackage{float}
\usepackage[table]{xcolor}
\usepackage{pgfplots}
\pgfplotsset{compat=newest}
\usepackage{caption}
\usepackage{subcaption}
\usepackage{fancyhdr}
\usepackage{blindtext}
\usepackage{natbib}
\setcitestyle{square}
\usepackage{url}
\usepackage{wrapfig}
\usepackage{amsmath}
\usepackage{parskip}
\usepackage{tabularx}
\newcolumntype{Y}{>{\centering\arraybackslash}X}
\usepackage{multirow}

\title{\vspace{-2cm}
\includegraphics[width=0.3\textwidth]{/Users/finn/Documents/Cardiff-University-logo-for-website}~\\[1cm]
Title
}
\author{Finnbar Wilson - 22031076}
\date{\today}

\begin{document}

\maketitle

\section*{Abstract}

\pagebreak
\section{Introduction}

NGC 3201 is a globular cluster, discovered by James Dunlop on the 28th of May 1826, located at 10$^{\text{h}}$ 17$^{\text{m}}$ 36.82$^{\text{s}}$/-46$^{\circ}$ 24' 44.9" (in RA/Dec) 5.0 kpc away from the Sun \citep{3201fact}. NGC 3201 has a large sub-cluster of black hole in its core making it an interesting source for observing the interactions in large populations of black holes \citep{blackholes}.

Globular clusters are among the oldest stellar populations in the universe, providing key insights into how stars and galactic structures evolve. In the Milky Way, some clusters are thought to have originated outside the galaxy due to their similar properties to satellite dwarf galaxies, whereas others are believed to have evolved within the milky way itself due to the observable effects of tidal forces and shocks in the inner galaxy. This allows for globular clusters to be classified by their characteristics as shown by \citet{Mackey} into three types: 'Young' halo(YH) which are thought to have been formed in external galaxies, 'Old' halo(OH) and 'Bulge/Disc'(BD) which are formed in the milky way. According to \citet{Mackey}, their study on globular clusters classified NGC 3201 as a YH cluster based on the metallicity and redder horizontal branch stars. NGC 3201 stands out from other clusters classified by \citet{Mackey} due to its irregular radial velocity and differential reddening across its face \citep{Kravtsov}. This makes it one of the few known clusters with an inhomogeneous stellar population for its size, which could affect how it has been classified.

In this report the structure of NGC 3201 will be analysed by finding the stellar populations inside the cluster and comparing them to isochrones to determine their age. These stellar populations will then allow a greater insight into the internal structer of the cluster creating a more actuate annalysis of its classification. Given that NGC 3201 is an abnormal cluster this report will also test various methods of determining the classification of abnormal clusters.

\section{Procedure}

\subsection{Calibration}

Two images of NGC 3201 were taken in the V and B filters on a 1.0m diameter telescope. Five stars were found in each filter to calibrate the zero point magnitude in each image and their data can be found in Table \ref{tab:calB} \& \ref{tab:calV}. To find the calibration stars a catalog of local stars from \citet{simbad} was overlaid in each image and 10 stars were selected in total and their known magnitudes recored. An apature photometery of each star was performed and recorded as well as their error.

\begin{table}[H]
\centering
\caption{Calibration stars in B filter}
\begin{tabular}{lccccc}
\toprule
ID & RA & Dec & B$_{\text{instrument}}$ & B$_{\text{simbad}}$ \\
\midrule
Cl* NGC 3201 CWFD 3-109 &  10:17:23.65 &  -46:24:17.31 & -12.566$\pm 0.004$ & 16.216 \\                  
Cl* NGC 3201 CWFD 3-198 &  10:17:34.49 &  -46:25:36.15 & -12.947$\pm 0.003$ & 15.800 \\                 
Cl* NGC 3201 CWFD 3-224 &  10:17:36.86 &  -46:23:11.97 & -15.062$\pm 0.001$ & 13.601 \\                 
NGC 3201 3401 &  10:17:38.12 &  -46:22:39.29 & -14.787$\pm 0.001$ & 14.080 \\                           
NGC 3201 4319 &  10:17:42.55 &  -46:27:15.42 & -14.812$\pm 0.001$ & 13.999 \\
\bottomrule
\end{tabular}
\caption*{ID is the stars identifcation searchable on the SIMBAD database, B$_{\text{instrument}}$ is the magnitude recorded in this experiment and B$_{\text{simbad}}$ is the known magnitude found on SIMBAD \citep{simbad}}
\label{tab:calB}
\end{table}

\begin{table}[h]
\centering
\caption{Calibration stars in V filter}
\begin{tabular}{lcccccc}
\toprule
ID & RA & Dec & V$_{\text{instrument}}$ & V$_{\text{simbad}}$\\
\midrule
2MASS J10173339-4620241 & 10:17:33.39 & -46:20:24.16 & -13.579$\pm 0.003$ & 15.650 \\                   
Cl* NGC 3201 CWFD 3-296 & 10:17:42.26 & -46:19:47.92 & -13.904$\pm 0.002$ & 15.230 \\                  
Cl* NGC 3201 CWFD 3-255 & 10:17:38.86 & -46:22:56.86 & -14.536$\pm 0.002$ & 14.730 \\                  
Cl* NGC 3201 CWFD 3-235 & 10:17:37.72 & -46:22:53.50 & -14.410$\pm 0.002$ & 14.910 \\                     
Cl* NGC 3201 CWFD 3-195 & 10:17:34.01 & -46:23:26.20 & -13.225$\pm 0.003$ & 16.030\\ 
\bottomrule
\end{tabular}
\caption*{ID is the stars identifcation searchable on the SIMBAD database, V$_{\text{instrument}}$ is the magnitude recorded in this experiment and V$_{\text{simbad}}$ is the known magnitude found on SIMBAD \citep{simbad}}
\label{tab:calV}
\end{table}

The equation to find the zero point magntidue is shown by Equation \ref{eq:zp}.
\begin{equation}
	m_{\text{zero point}} = m_{\text{simbad}} - m_{\text{instrument}}
	\label{eq:zp}
\end{equation}
Where $m$ is the magnitude (B or V). This produced a zero point magnitude of: B$_{\text{zero point}} = 28.774 \pm 0.005$ and V$_{\text{zero point}} = 29.2407 \pm 0.0010$. The errors associated with these values are from the error in the $m_{\text{instrument}}$ recordings as well as the error in the $m_{\text{simbad}}$ which are not shown in Table \ref{tab:calB} \& \ref{tab:calV} but can be found on \citet{simbad}. 

\subsection{Automated detection}

An object detection tool was used to quickly find the magnitudes of large number of stars in each image. 

\subsection{Manual detection}

\section{Analysis}

\section{Discussion}

\section{Conclusion}

\bibliographystyle{agsm}
\bibliography{Mainref.bib}

\end{document}